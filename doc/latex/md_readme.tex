Cubium is a free and open-\/source flight software for Linux-\/based spacecraft systems. Cubium allows for a more standardized and streamlined method of handling systems with many connected components by providing the neccesarry network to allow automatic discovery and communication between components. Developed with undergraduate Cube\+Sat teams using systems such as Beaglebone Blacks and Raspberry Pis in mind, Cubium\textquotesingle{}s purpose is to lower the bar of entry for satellite development.

Cubium is designed using the Space Plug-\/and-\/play Architecture (S\+PA), a specification for a kind of modular satellite software architecture. It has a proven mission-\/success track record on Air Force and Space Dynamics Laboratory payloads.

For a fun introduction on the inner workings of Cubium, see \href{https://drive.google.com/file/d/0ByiGNyJUAlpISUo5WDFwSkh3YU0/view?usp=sharing}{\tt this illustrated writeup.}

For a very detailed look into the machinations of S\+PA in general, see \href{http://digitalcommons.usu.edu/etd/1422/}{\tt Jacob Holt Christensen\textquotesingle{}s dissertation.}

\subsection*{Project Status}


\begin{DoxyItemize}
\item {\bfseries Version Alpha 1.\+2.\+0}
\begin{DoxyItemize}
\item Support for sending strings between components
\end{DoxyItemize}
\item {\bfseries Version Alpha 1.\+1.\+0}
\begin{DoxyItemize}
\item Finalized architecture for software demo shown at Small\+Sat conference
\end{DoxyItemize}
\item {\bfseries Version Alpha 1.\+0.\+0}
\begin{DoxyItemize}
\item All necessary framework completed for support of basic component systems.
\end{DoxyItemize}
\item {\bfseries Version Alpha 0.\+0.\+6}
\begin{DoxyItemize}
\item Successful transmission of S\+PA messages across processes via socket communication
\end{DoxyItemize}
\item {\bfseries Version Alpha 0.\+0.\+5}
\begin{DoxyItemize}
\item Major backend refactoring of S\+PA Messages
\end{DoxyItemize}
\item {\bfseries Version Alpha 0.\+0.\+4}
\begin{DoxyItemize}
\item Added a basic subscription service
\begin{DoxyItemize}
\item Direct component-\/to-\/component subscription
\item Non-\/prioritized publishing
\end{DoxyItemize}
\item Improvements to message handling
\item Additional tests for components
\end{DoxyItemize}
\item {\bfseries Version Alpha 0.\+0.\+3}
\begin{DoxyItemize}
\item Basic implementations of\+:
\begin{DoxyItemize}
\item Local Spa\+Messages
\item Components
\item Local Subnet Manager
\end{DoxyItemize}
\end{DoxyItemize}
\item {\bfseries Version Alpha 0.\+0.\+2}
\begin{DoxyItemize}
\item Debian dev environment complete
\end{DoxyItemize}
\item {\bfseries Version Alpha 0.\+0.\+1}
\begin{DoxyItemize}
\item Planning A\+PI\textquotesingle{}s and project planning
\end{DoxyItemize}
\end{DoxyItemize}

\subsection*{Getting Started}

\subsubsection*{Developer Tools}

Cubium relies on a handful of developer tools. The following is a list of things we\textquotesingle{}ll be using\+:
\begin{DoxyItemize}
\item Vagrant -\/ Virtual development environment
\item Git -\/ Version control system
\item Google Test -\/ Unit testing framework
\item Doxygen -\/ Documentation generator
\item C\+Make -\/ Build system automation
\end{DoxyItemize}

\subsubsection*{Set up Vagrant}

Cubium uses Vagrant to create a development environment to match the devices that Cubium will run on. It also eliminiates \char`\"{}well, it works on my system\char`\"{} bugs.

{\bfseries For instructions on getting the dev environment up and running, see the \href{https://github.com/Cubium/Cubium/wiki/Setting-up-the-Cubium-Dev-Environment}{\tt wiki page}}

\subsubsection*{Build Project}

\paragraph*{TL;DR}


\begin{DoxyItemize}
\item Run C\+Make in project directory {\ttfamily cmake .}
\item Run generated makefile {\ttfamily make \mbox{[}optional-\/target\mbox{]}}
\end{DoxyItemize}

Cubium uses C\+Make for a build system. Makefiles are generally platform-\/dependent, so C\+Make generates a different Makefile for each system in order to allow for cross-\/plaform functionality.

\subsubsection*{Build Docs}

\paragraph*{TL;DR}


\begin{DoxyItemize}
\item Run Doxygen with project doxyfile {\ttfamily doxygen ./\+Doxyfile}
\item View your docs. They should now live in {\ttfamily docs/}
\end{DoxyItemize}

Cubium uses the documentation generator Doxygen to build documentation. Annotated source code is parsed by Doxygen to generate La\+TeX and H\+T\+ML files.

Doxygen is configured with a file titled {\ttfamily Doxyfile}.


\begin{DoxyItemize}
\item Build Documentation
\begin{DoxyItemize}
\item Invoke commandline tool
\begin{DoxyItemize}
\item {\ttfamily doxygen Doxyfile}
\end{DoxyItemize}
\end{DoxyItemize}
\end{DoxyItemize}

This will read all configuration options from the Doxyfile, find and parse the source code, and generate the documentation.

If the documentation is successfully built, there should be a new directory title {\ttfamily docs/} that should contain both H\+T\+ML and La\+TeX documentation.


\begin{DoxyItemize}
\item Read Docs
\begin{DoxyItemize}
\item Open up {\ttfamily docs/html/index.\+html} in your web browser to browse docs
\end{DoxyItemize}
\end{DoxyItemize}

\subsubsection*{Running Tests}

Cubium tests use the Google Test testing framework for unit testing. Test test test.
\begin{DoxyItemize}
\item To run test suite\+:
\begin{DoxyItemize}
\item Generate a makefile with C\+Make {\ttfamily cmake .}
\item Build tests with makefile {\ttfamily make run\+Tests}
\item Run test executable {\ttfamily ./run\+Tests}
\end{DoxyItemize}
\end{DoxyItemize}

\subsubsection*{Testing}

Cubium uses Google Test for unit testing and C\+Make for a build system. The short version of running tests is this\+:

Classes should be kept small and have functioning unit tests. When adding a new header file for a class, a header file of the same name should be added to the {\ttfamily test/} directory.

To add a new class to the project\+:
\begin{DoxyItemize}
\item Create header file {\ttfamily my\+\_\+class\+\_\+name.\+hpp} (File names should be snake case -\/ lowercase words seperated with underscores)
\begin{DoxyItemize}
\item Define class ```cpp \#ifndef M\+Y\+\_\+\+C\+L\+A\+S\+S\+\_\+\+N\+A\+M\+E\+\_\+\+H\+PP \#define M\+Y\+\_\+\+C\+L\+A\+S\+S\+\_\+\+N\+A\+M\+E\+\_\+\+H\+PP class My\+Class\+Name\{\}; \#endif ```
\begin{DoxyItemize}
\item Must have include guards
\item Class name should be Upper\+Camel\+Case, where each first letter of a words is capitalized. Including the first word.
\end{DoxyItemize}
\end{DoxyItemize}
\item Add new testing file {\ttfamily test/my\+\_\+class\+\_\+name.\+hpp}
\item Write tests for your class ```cpp \#include \char`\"{}../path/to/my\+\_\+class\+\_\+name.\+hpp\char`\"{}

T\+E\+S\+T(\+My\+Class\+Name, my\+Method)\{ My\+Class\+Name my\+Class; E\+X\+P\+E\+C\+T\+\_\+\+EQ(my\+Class.\+my\+Method(),0); \} ```
\begin{DoxyItemize}
\item Be sure to include class header in test file
\end{DoxyItemize}
\item Include your test header in main test file
\begin{DoxyItemize}
\item Open {\ttfamily test/gtest\+\_\+main.\+cpp}
\item Include your new test header file
\end{DoxyItemize}
\item Hooray! Now you can run your tests! \+:D
\end{DoxyItemize}

\subsubsection*{Documentation}

Cubium uses Doxygen to build documentation from source code. This means that one can add comments with a special format in the code so that Doxygen may build pretty H\+T\+ML docs that can be referenced by all other developers and users.

Here is an example of what this might look like to document a function. 
\begin{DoxyCode}

\textcolor{keywordtype}{bool} example(\textcolor{keywordtype}{int} myParam)\{\textcolor{keywordflow}{return} \textcolor{keyword}{true};\}
\end{DoxyCode}
 